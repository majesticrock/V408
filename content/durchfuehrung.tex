\section{Durchführung}
\label{sec:Durchführung}
Im ersten Versuchsteil soll sowohl das Abbildungsgesetz, als auch die Linsengleichung verifiziert werden. Dazu werden auf einer optischen 
Bank eine Lichtquelle (hier: eine Halogenlampe), der Gegenstand "Pearl L", eine zu untersuchende dünne Linse, sowie ein Schirm positioniert.
Es werden nun die Abmessungen des Gegenstandes $G$ und die Gegenstandsweite $g$ bestimmt. Der Schirm wird so verschoben, dass ein scharfes 
Bild abgebildetw wird. Es werden die Bildgröße $B$ und die Bildweite $b$ notiert. Die Gegenstandsweite $g$ wird nun verändert und die zugehörige
Bildgröße $B$ beziehunsgweise Bildweite $b$ nach erneutem Einstellen des Schirmes notiert. So wird die Messung insgesamt zehnmal durchgeführt.

In einem weiteren Versuchsteil wird die Brennweite $f$ mittels der Methode von Bessel, wie in \autoref{sec:theorie} beschrieben, bestimmt.
Hierzu werden Schirm und Gegenstand in einem festen Abstand $e$ voneinander aufgestellt. Dieser Abstand wird notiert. Nun werden die beiden
Positionen der Linse bestimmt, an denen das Bild am Schirm scharf erscheint. Die entsprechenden Gegenstandsweiten $g$ und Bildweiten $b$
werden für beide Positionen notiert. Es werden insgesamt zehn Messungen für verschiedene Abstände $e$ aufgenommen.  

Zur Untersuchung der chromatischen Abberation wird die Methode von Bessel analog zum vorherigen Versuchsteil verwendet. Hierbei werden nun 
zu je fünf verschiedenen Abständen $e$ die Messung für einen blauen beziehunsgweise roten Filter, welcher vor dem Gegenstand befestigt wird,
durchgeführt und die Werte der Gegenstandsweite $g$ und Bildweite $b$ für die jeweiligen beiden Positionen notiert.

Im letzten Versuchsteil wird die Methode von Abbe, wie sie in \autoref{sec:theorie} beschrieben wird, verwendet. Hierbei werden eine Sammellinse
und eine Zerstreuungslinse nahe zusammen gestellt, dass sich die Füße der Apperaturen berühren. Zu beachten ist, dass die Sammellinse
eine Brennweite von $f = 100$ und die Zerstreuungslinse eine Brennweite von $f = -100$ hat. Vom Schirm aus gesehen befindet sich die
Sammellinse vor der Zerstreuungslinse, wie es in \autoref{fig:zusammen} dargestellt ist. Beim Verschieben der Linsen muss der Abstand
zwischen den Linsen konstant bleiben. Sie werden nun so verschoben, dass sich ein scharfes Bild auf dem Schirm ergibt. Es wird ein Punkt $A$
bei einer der Linsen festgelegt von dem aus die Bildweite $b'$ und die Gegenstandsweite $g'$ bestimmt werden. Der Punkt $A$ befindet sich
in dieser Messung in der Mittelebene der Sammellinse. Des Weiteren wird die Bildgröße $B$ gemessen. Diese Messung wird für zehn verschiedene
Schirmpositionen durchgeführt.