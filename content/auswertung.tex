\section{Auswertung}
\label{sec:Auswertung}

Zunächst werden die Linsengleichung GLEICHUNG und das Abbildungsgesetz GLEICHUNG verifiziert.
Die hierzu verwendeten Daten sind in \autoref{tab:schnittpunkt} zu finden.
Dabei ist $V_1 = \frac{B}{G}$ und $V_2 = \frac{b}{g}$.

\begin{table}[!htp]
\centering
\caption{Daten der Messung zur Vschnittpunkt-zoomerifizierung der Linsengleichung.}
\label{tab:schnittpunkt}
\begin{tabular}{S[table-format=2.1] S[table-format=2.1] S[table-format=1.1] S[table-format=1.1] S[table-format=1.1] S[table-format=2.1] S[table-format=1.2]}
\toprule
{$g$ / cm} & {$b$ / cm} & {$B$ / cm} & {$V_1$} & {$V_2$} & {$\frac{| V_1 - V_2 |}{V_1}$ / \%} & {$f$ / cm} \\
\midrule
18.0 & 22.0 & 3.5 & 1.1 & 1.2 &  4.8 & 9.90 \\
19.0 & 20.4 & 3.1 & 1.0 & 1.1 &  3.9 & 9.84 \\
21.8 & 18.2 & 2.5 & 0.8 & 0.8 &  0.2 & 9.92 \\
23.0 & 17.5 & 2.2 & 0.7 & 0.8 &  3.8 & 9.94 \\
24.0 & 16.7 & 1.9 & 0.6 & 0.7 &  9.9 & 9.85 \\
26.0 & 15.8 & 1.8 & 0.6 & 0.6 &  1.3 & 9.83 \\
28.0 & 15.3 & 1.5 & 0.5 & 0.5 &  9.3 & 9.89 \\
30.0 & 14.6 & 1.4 & 0.5 & 0.5 &  4.3 & 9.82 \\
32.0 & 14.1 & 1.2 & 0.4 & 0.4 & 10.2 & 9.79 \\
34.0 & 13.7 & 1.2 & 0.4 & 0.4 &  1.7 & 9.77 \\
\bottomrule
\end{tabular}
\end{table}

Der mittlere Abweichung der Differenzen von $V_1$ und $V_2$ ergibt sich zu

\begin{center}
    $\overline{\Delta V} = \frac{1}{N} \sum^N_{i=1} \frac{| V_{1,i} - V_{2,i} |}{V_{1,i}} = (4,8 \pm 0,4) \, \%$.
\end{center}

Die Brennweite der Linse kann einerseits bestimmt werden, in dem die Werte aus \autoref{tab:schnittpunkt} gemittelt werden.
Andererseits können die $b_i$ auf die $y$-Achse und die $g_i$ auf die $x$-Achse gelegt werden. Anschließend werden die jeweiligen Wertepaare mit einer Gerade verbunden.
Der Schnittpunkt dieser entspricht dann der Brennweite.
Der so entstehende Plot ist in \autoref{fig:schnittpunkt} und \autoref{fig:schnittpunkt-zoom} zu sehen.

\begin{figure}
  \centering
  \includegraphics[width=0.95\textwidth]{build/plot_schnittpunkt.pdf}
  \caption{Plot der Messwerte von $g$ und $b$ jeweils auf den Achsen, wobei die einzelnen Wertepaare miteinander verbunden sind.}
  \label{fig:schnittpunkt}
\end{figure}

\begin{figure}
  \centering
  \includegraphics[width=0.95\textwidth]{build/plot_schnittpunkt_zoom.pdf}
  \caption{\autoref{fig:schnittpunkt} in einem kleineren Ausschnitt zum besseren Ablesen des Schnittpunktes der Geraden.}
  \label{fig:schnittpunkt-zoom}
\end{figure}

Damit ergibt sich für die Brennweite

\begin{align*}
    f_text{Mittel} &= (9,854 \pm 0,009) \, \symup{cm}\\
    f_\text{Plot} &= 10 \, \symup{cm}.
\end{align*}