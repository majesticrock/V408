\section{Auswertung}
\label{sec:Auswertung}

\subsection{Verifizierung der Linsengleichung und des Abbildungsgesetzes}

%TODO: Gleichung
Zunächst werden die Linsengleichung GLEICHUNG und das Abbildungsgesetz GLEICHUNG verifiziert.
Die hierzu verwendeten Daten sind in \autoref{tab:schnittpunkt} zu finden.
Dabei ist $V_1 = \frac{B}{G}$ und $V_2 = \frac{b}{g}$.

\begin{table}[!htp]
\centering
\caption{Daten der Messung zur Verifizierung der Linsengleichung.}
\label{tab:schnittpunkt}
\begin{tabular}{S[table-format=2.1] S[table-format=2.1] S[table-format=1.1]}
\toprule
{$g$ / cm} & {$b$ / cm} & {$B$ / cm} \\
\midrule
18.0 & 22.0 & 3.5 \\
19.0 & 20.4 & 3.1 \\
21.8 & 18.2 & 2.5 \\
23.0 & 17.5 & 2.2 \\
24.0 & 16.7 & 1.9 \\
26.0 & 15.8 & 1.8 \\
28.0 & 15.3 & 1.5 \\
30.0 & 14.6 & 1.4 \\
32.0 & 14.1 & 1.2 \\
34.0 & 13.7 & 1.2 \\
\bottomrule
\end{tabular}
\end{table}

Der mittlere Abweichung der Differenzen von $V_1$ und $V_2$ ergibt sich zu

\begin{center}
    $\overline{\Delta V} = \frac{1}{N} \sum^N_{i=1} \frac{| V_{1,i} - V_{2,i} |}{V_{1,i}} = (4,8 \pm 0,4) \, \%$.
\end{center}

Die Brennweite der Linse kann einerseits bestimmt werden, in dem die Werte aus \autoref{tab:schnittpunkt} gemittelt werden.
Andererseits können die $b_i$ auf die $y$-Achse und die $g_i$ auf die $x$-Achse gelegt werden. Anschließend werden die jeweiligen Wertepaare mit einer Gerade verbunden.
Der Schnittpunkt dieser entspricht dann der Brennweite.
Der so entstehende Plot ist in \autoref{fig:schnittpunkt} und \autoref{fig:schnittpunkt-zoom} zu sehen.

\begin{figure}
  \centering
  \includegraphics[width=0.95\textwidth]{build/plot_schnittpunkt.pdf}
  \caption{Plot der Messwerte von $g$ und $b$ jeweils auf den Achsen, wobei die einzelnen Wertepaare miteinander verbunden sind.}
  \label{fig:schnittpunkt}
\end{figure}

\begin{figure}
  \centering
  \includegraphics[width=0.95\textwidth]{build/plot_schnittpunkt_zoom.pdf}
  \caption{\autoref{fig:schnittpunkt} in einem kleineren Ausschnitt zum besseren Ablesen des Schnittpunktes der Geraden.}
  \label{fig:schnittpunkt-zoom}
\end{figure}

Damit ergibt sich für die Brennweite

\begin{align*}
    f_\text{Mittel} &= (9,854 \pm 0,009) \, \symup{cm}\\
    f_\text{Plot} &= (10 \pm 1) \, \symup{cm}.
\end{align*}



\subsection{Bestimmung der Brennweite einer Linse nach der Methode von Bessel}

%TODO: Gleichung
Zur Bestimmung der Brennweite nach Bessel werden die Daten in \autoref{tab:bessel} verwendet.
Die Brennweiten $f_i$, welche zusätzlich zu den Messwerten in der Tabelle befinden, berechnen sich dabei nach GLEICHUNG.
Zur Berechnung von $d$ wird der Mittelwert aus $b_1$ und $g_2$ beziehungsweise jener aus $b_2$ und $g_1$ verwendet.

\begin{table}[!htp]
\centering
\caption{Daten der Messung zur Bestimmung der Brennweite einer Linse nach Bessel bei weißem Licht.}
\label{tab:bessel}
\begin{tabular}{S[table-format=2.0] S[table-format=2.1] S[table-format=2.1] S[table-format=2.1] S[table-format=2.1] S[table-format=2.1] S[table-format=1.2]}
\toprule
{$e$ / cm} & {$g_1$ / cm} & {$b_1$ / cm} & {$g_2$ / cm} & {$b_2$ / cm} & {$d$ / cm} & {$f$ / cm} \\
\midrule
40 & 17.9 & 22.1 & 23.0 & 17.0 &  5.1 & 9.84 \\
42 & 15.9 & 26.1 & 26.2 & 15.8 & 10.3 & 9.87 \\
44 & 14.9 & 29.1 & 29.0 & 15.0 & 14.1 & 9.87 \\
46 & 12.4 & 31.6 & 32.0 & 14.0 & 18.6 & 9.62 \\
48 & 14.1 & 33.9 & 34.4 & 13.6 & 20.3 & 9.85 \\
50 & 13.6 & 36.4 & 36.5 & 13.5 & 22.9 & 9.88 \\
52 & 13.4 & 38.6 & 40.1 & 11.9 & 26.7 & 9.57 \\
54 & 13.1 & 40.9 & 41.1 & 12.9 & 28.0 & 9.87 \\
56 & 13.0 & 43.0 & 43.3 & 12.6 & 30.4 & 9.89 \\
58 & 12.8 & 45.2 & 45.5 & 12.5 & 32.7 & 9.89 \\
\bottomrule
\end{tabular}
\end{table}

Aus diesen Werten wird der Mittelwert gebildet, um einen Wert für $f$ zu erhalten:

\begin{center}
  $f = (9,815 \pm 0,008) \, \symup{cm}$.
\end{center}