\section{Diskussion}
\label{sec:Diskussion}

Im ersten Versuchsteil weichen die einzelnen Vergrößerungsmaßstäbe lediglich um $4,8\%$ voneinander ab.
Der Schnittpunkt der Geraden in den Plots ist jedoch in einem Bereich von circa $(10 \pm 1)$ cm verteilt.
Die beiden für die Brennweite bestimmten Werte

\begin{align*}
    f_\text{Mittel} &= (9,854 \pm 0,009) \, \symup{cm}\\
    f_\text{Plot} &= (10 \pm 1) \, \symup{cm}.
\end{align*}

weichen nur sehr gering von dem Herstellerwert von $10 \, \symup{cm}$ ab.
Der rechnerisch bestimmte hat eine Abweichung von $1,5 \%$. Der aus dem Plot abgelesene Wert hat im Mittel keine Abweichung, ist jedoch mit einem großen Fehler behaftet.

Bei der Methode nach Bessel werden folgende Werte bestimmt.

\begin{align*}
  f_\text{Weiß} &= (9,815 \pm 0,008) \, \symup{cm}\\
  f_\text{Rot} &=  (9,913 \pm 0,009) \, \symup{cm}\\
  f_\text{Blau} &= (9,943 \pm 0,005) \, \symup{cm}.
\end{align*}

Es fällt auf, dass die Abweichung von dem Theoriewert bei weißen Licht mit $1,9 \%$ ebenfalls sehr gering ist.
Die chromatische Abberation scheint lediglich einen geringen Beitrag zu haben, da sich beide Werte nur um $0,3 \%$ voneinander unterscheiden.

Es ist jedoch zu bemerken, dass insbesondere bei kleinen Bildgrößen und großen Bildweiten die Einstellung der Apperatur sehr schwierig ist.
Aufgrund großer Distanzen ist das Licht auf dem Schirm kaum zu erkennen, ebenso lässt sich nur ungefähr definieren, wann genau die Abbildung scharf ist.
Des Weiteren sind die Linsen nicht perfekt geformt, sodass es eine Abberation von achsenfernen Strahlen gibt, die das Ergebnis weiterhin verfälschen.
Auch war der Raum, in dem der Versuch durchgeführt wurde mit externen Lichtquellen beleuchetet, wodurch das Licht des eigentlichen Versuches umso schwächer erscheint.
Des Weiteren wird die Bildgröße mit einem Geodreieck vermessen, wodurch auch hier eine gewisse Ungenauigkeit entsteht.
Mit diesen Ungenauigkeiten können die Abweichungen der Werte gut erklärt werden.

Die Ausgleichsgeraden bei der Methode nach Abbe nähern die Messwerte gut an.
Es ist jedoch auffällig, dass sich die berechneten Brennweiten 

\begin{align*}
  f_\text{g} &= (14,6 \pm 0,2) \, \symup{cm}\\
  f_\text{b} &= (21,7 \pm 0,5) \, \symup{cm}
\end{align*}

sehr stark voneinander unterscheiden. Dies lässt sich jedoch ebenfalls mit den oben genannten Ungenauigkeiten erklären.
Der Abstand von Lichtquelle und Schirm ist in diesem Versuchsteil besonders groß, ebenso sind die entstehenden Bilder sehr klein, wodurch die Ungenauigkeiten umso größer werden.

Die angegebenen Werte der Linsen sind je $f_1 = 10$ cm und $f_2 = -10$ cm. Die Brennweite eines Systems zweier solcher dünner Linsen mit dem Abstand $d = 6$ cm beträgt

\begin{center}
  $f = - \frac{f_1 \cdot f_2}{d} = 16,7$ cm.
\end{center}

Der Mittelwert der bestimmen Brennweiten

\begin{center}
  $f_\text{Mittel} = (19 \pm 3)$ cm.
\end{center}

weicht um $14 \%$ von dem Theoriewert ab. Auch diese Abweichung kann mit den oben genannten Ungenauigkeiten erklärt werden.